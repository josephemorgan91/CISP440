\documentclass[12pt,titlepage]{article}
\usepackage[margin=1in]{geometry}
\usepackage{multicol}
\usepackage{fancyhdr}
\usepackage{listings}
\usepackage{amsmath}
\usepackage{graphicx}
\setlength\parindent{0pt}
\fancyhf{}
\rfoot{Page \thepage}

\begin{document}
\begin{titlepage}
	\centering
	\vfill
	{\bfseries\Large
		Joseph Morgan\\
		\large
		Homework 15\\
		\vskip2cm
		CISP440\\
	}
	\vfill
	\vfill
	\vfill
\end{titlepage}
\section*{Section 10.3}
\subsection*{\textit{In Exercises 1-6, determine whether the givin grammar is
context-sensitive, context-free, regular or nonoe of these. Give all
characterizations that apply.}}
\subsection*{2. $T = \{a, b, c\}, N = \{\sigma, A, B\}$, with productions}
\begin{multicols}{3}
	$\sigma \rightarrow b\sigma$,\\
	$\sigma \rightarrow aA$,\\
	\columnbreak\\
	$AB \rightarrow BA$,\\
	$A \rightarrow bA$,\\
	\columnbreak\\
	$A \rightarrow a$,\\
	$\sigma \rightarrow b$,\\
\end{multicols}
\subsection*{and $\sigma$ as a starting point.}
This grammar is context context-sensitive, as evidenced by the production that
has more than one symbol on the left-hand-side.
\subsection*{3. $T = \{a, b\}, N = \{\sigma, A, B\}$, with productions}
\begin{multicols}{3}
	\sigma \rightarrow A,\\
	\sigma \rightarrow AAB,\\
	Aa \rightarrow ABa,
	\columnbreak\\
	A \rightarrow aa,\\
	Bb \rightarrow ABb,
	\columnbreak\\
	AB \rightarrow ABB,\\
	B \rightarrow b
\end{multicols}
\subsection*{and $\sigma$ as a starting point.}
This grammar is also context-sensitive, because of the $Bb \rightarrow ABb$,
$Aa \rightarrow ABa$ and $AB \rightarrow ABB$ productions.
\subsection*{\textit{In Exercises 7-11, show that the given string $\alpha$ is in
L(G) for the given grammer G by giving a definition of $\alpha$}}
\subsection*{8. \textit{abab}, Exercise 2}
$\sigma \rightarrow AB$\\
$AB \rightarrow aAB$\\
$aAB \rightarrow aABb$\\
$aABb \rightarrow aBAb$\\
$aBAb \rightarrow abAb$\\
$abAb \rightarrow abab$\\
\subsection*{11. \textit{abaabbabba}, Exercise 5}
$<S> \rightarrow a<A> \rightarrow ab<B> \rightarrow aba<S> \rightarrow
abaa<A> \rightarrow abaab<B>\\ \rightarrow abaabb<A> \rightarrow abaabba<S>
\rightarrow abaabbab<S> \rightarrow abaabbabb<S> \rightarrow abaabbabba$
\subsection*{12. Write the grammars of Examples 10.3.4 and 10.3.9 and Exercises 1-4 and 6 in BNF}
\bf{\textit{10.3.4:}}\\ 
$<\sigma> ::= b<\sigma>|a<S>\\
<S> ::= b<S> | b$\\
\bf{\textit{10.3.9:}}\\
$<\sigma> ::= a<A><B>|a<B>\\
<A> ::= a<A><C>|a<C>\\
<B> ::= <D>c\\
<D> ::= b\\
<C><D> ::= <C><E>\\
<C><E> ::= <D><E>\\
<D><E> ::= <D><C>\\
<C>c ::= <D>cc\\$
\bf{\textit{Exercise 1}}\\
$
<\sigma>::=b<\sigma>|a<A>|b\\
<A>::=b<A>|a<\sigma>|a\\
$
\bf{\textit{Exercise 2}}\\
$
<\sigma>::=<A><B>|a\\
<A><B>::=<B><A>\\
<A>::=a\sigma|b<A>|a\\
$
\bf{\textit{Exercise 3}}\\
$
<\sigma> ::= <A>|<A><A><B>\\
<A>a::=<A><B>a\\
<A>::=aa\\
<B>b::=<A><B>b\\
<A><B>::=<A><B><B>\\
<B>::=b\\
$
\bf{\textit{Exercise 4}}\\
$
<\sigma>::=<B><A><B>|<A><B><A>\\
<A>::=<A><B>|a<A>|ab\\
<B>::=<B><A>|b\\
$
\bf{\textit{Exercise 6}}\\
$
<\sigma>::=<A><A>\sigma\\
<A><A>::=<B>\\
<B>::=b<B>\\
<A>::=a\\
$
\subsection*{\textit{In Exercises 15-24, write a grammar that generates the
strings having the givin property.}}
\subsection*{16. Strings over $\{a, b\}$ starting with $a$.}
$T = \{a, b\}, N = \{\sigma, A, B\}\\
<\sigma>::=<A><B>|<B>\\
<A>::=a|b\\
<B>::=ab\\
$
\subsection*{17. Strings over $\{a, b\}$ containing $a$.}
$T = \{a, b\}, N = \{\sigma, A, B, C\}\\
<\sigma>::=<\sigma><C>|<C><\sigma>\\
<A>::=a|b\\
<B>::=ab\\
<C>::=<A><B>|<B><A>
$
\subsection*{19. Integers with no leading 0's.}
$T = \{0,1,2,3,4,5,6,7,8,9\}, N = \{\sigma, NONZERO, INT\}\\
<\sigma>::=<NONZERO>|<NONZERO><INT>|0\\
<NONZERO>::=1,2,3,4,5,6,7,8,9\\
<INT>::=0,1,2,3,4,5,6,7,8,9\\
$
\subsection*{21. Exponential numbers.}
$T = \{0-9, ., E, -, +\}, N = \{\sigma, I, B, S\}\\
<\sigma>::=0|<B>|<B>E<I>\\
<I>::=<S>0-9|<S><I>0-9\\
<B>::=<I>|<I>.<I>|\\
<S>::=+|-|\lambda\\
$
\subsection*{28}
Yes, any string generated would have equal numbers of a's and b's. You can tell
because the only productions each add one a and one b, just in different places
in the string.
\end{document}
